% -*- root: main.tex -*-
% 1) pdflatex main
% 2) makeindex main.idx -s StyleInd.ist
% 3) biber main
% 4) pdflatex main x 2
%-------------------------------------------------------------------------------
\chapterimage{chapter_head_1.pdf} 

%-------------------------------------------------------------------------------
\chapter{ROS 도구}

%-------------------------------------------------------------------------------
\section{ROS 도구 (RVIZ)}\index{ROS 도구 (RVIZ)}

%-------------------------------------------------------------------------------
\subsection{도구}\index{도구}

%-------------------------------------------------------------------------------

\begin{enumerate}
\item The first item
\item The second item
\item The third item
\end{enumerate}

\begin{enumerate}
  \setcounter{enumi}{4}
  \item fifth element
\end{enumerate}

\begin{enumerate}
  \item The first item
  \begin{enumerate}
    \item Nested item 1
    \item Nested item 2
  \end{enumerate}
  \item The second item
  \item The third etc \ldots
\end{enumerate}


\begin{itemize}[leftmargin=*]
\item The first item
\item The second item
\item The third item
\end{itemize}

\begin{corollary}[Corollary name]
asdfasdfasdfasdfasdf
\end{corollary}

\begin{description}
\item[Name] Description
\item[Word] Definition
\item[Comment] Elaboration
\end{description}


\begin{description}
  \item[First] \hfill \\
  The first item
  \item[Second] \hfill \\
  The second item
  \item[Third] \hfill \\
  The third etc \ldots
\end{description}

\begin{table}[h]
\centering
\begin{tabular}{l l l}
\toprule
\textbf{Treatments} & \textbf{Response 1} & \textbf{Response 2}\\
\midrule
Treatment 1 & 0.0003262 & 0.562 \\
Treatment 2 & 0.0015681 & 0.910 \\
Treatment 3 & 0.0009271 & 0.296 \\
\bottomrule
\end{tabular}
\caption{Table caption}
\end{table}

%------------------------------------------------

\section{Figure}\index{Figure}

\begin{figure}[h]
\centering
\includegraphics[scale=0.5]{picture}
\caption{caption}
\end{figure}

\begin{figure}[h]
\centering
\includegraphics[width=0.5\columnwidth]{picture}
\caption{caption}
\end{figure}



\vspace{\baselineskip}

\setcounter{num}{0}

\vspace{\baselineskip}
\noindent
\stepcounter{num}
\thenum


\setcounter{num}{0}
\vspace{\baselineskip}
\noindent
\stepcounter{num}\circled{\thenum} USB 카메라를 컴퓨터에 연결한다.
\stepcounter{num}\circled{\thenum} 카메라 정보를 확인한다.


→←↑↓

\textcolor{green}{녹색}
{\color{limegreen}녹색}
\textcolor{orange}{주황색}
\textcolor{red}{적색}


\vspace{\baselineskip}
\begin{lstlisting}[language=ROS]
$ roscore
\end{lstlisting}



\cite{book_key}

\textbf{섹션~\ref{sec:RosTerm}~\nameref{sec:RosTerm}(pp.\pageref{sec:RosTerm})}

\textasciitilde

\begin{center} 
ROS은 \textbf{메타운영체제(Meta-Operating System)}이다.
\end{center}

%-------------------------------------------------------------------------------
%-------------------------------------------------------------------------------
%-------------------------------------------------------------------------------
%-------------------------------------------------------------------------------
%-------------------------------------------------------------------------------
%-------------------------------------------------------------------------------
%-------------------------------------------------------------------------------
%-------------------------------------------------------------------------------
%-------------------------------------------------------------------------------
%-------------------------------------------------------------------------------
%-------------------------------------------------------------------------------
%-------------------------------------------------------------------------------
%-------------------------------------------------------------------------------
%-------------------------------------------------------------------------------
%-------------------------------------------------------------------------------
