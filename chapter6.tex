% -*- root: main.tex -*-

%-------------------------------------------------------------------------------
\chapterimage{chapter_head_1.pdf} 

%-------------------------------------------------------------------------------
\chapter{ROS 도구}

%-------------------------------------------------------------------------------
\section{ROS 도구 (RVIZ)}\index{ROS 도구 (RVIZ)}

%-------------------------------------------------------------------------------
\subsection{도구}\index{도구}

ROS는 "로봇 운영체제 강좌 : ROS 명령어" 에서 설명한 커맨드형 명령어 이외에도 ROS 활용에 필요한 다양한 도구들이 존재한다. 이는 커맨드형 명령어와 상호보완하는 형태로 ROS 유저들에게는 필수적으로 알아둬야 할 부분이다.

ROS 도구로는 ROS 유저들이 개인적으로 공개한 툴까지 포함하여 상당히 많은 수가 있는데 그 중에서 이번 강좌에서는 1) 3차원 시각화를 돕는 RViz, 2) 데이터 로깅 툴인 rqt\_bag, 3) 데이터 플롯 툴인 rqt\_plot. 4) 노드간의 관계 및 메시지를 확인 가능한 rqt\_graph , 5) GUI 통합 툴인 rqt 을 알아볼 것이다. 있다. 이들 툴들은 ros와의 직접적인 처리를 행하는 것은 아니지만 ROS를 이용한 프로그래밍에 매우 유용한 보조툴이다. 

특히, ROS Fuerte 버전부터는 RViz는 rqt 라는 이름으로 rqt\_bag, rqt\_plot, rqt\_graph 등과 함께 36가지 플러그인이 통폐합되어 종합 GUI 툴로써 사용 가능해졌다. 그뿐만 아니라 rqt는 QT로 개발되어 있기 때문에 유저들이 자유롭게 플러그인을 개발하여 추가할 수도 있어서 매우 편리하다. 이번 강좌에서는 아래의 내용에 대해서 알아보도록 하자.

\begin{itemize}[leftmargin=*]
\item RViz : 3D visualization tool / 3차원 시각화 툴
\item rqt\_bag : Logging and Visualization Sensor Data, rosbag gui tool / 메시지 기록 GUI 유틸리티
\item rqt\_plot : Data Plot Tool / 2차원 데이터 플롯 툴 
\item rqt\_graph : 노드 및 메시지간의 상관 관계를 그래프로 나타내는 툴
\item rqt : QT 기반의 ROS GUI 개발 툴
\end{itemize}





































%-------------------------------------------------------------------------------